\begin{table*}
  \footnotesize
  \centering
  \renewcommand{\arraystretch}{2.5}
  \begin{tabular*}{\textwidth}{l l l l}
     & \pbox{1.1 in}{\emph{NVRAM\newline Disk-Replacement}} & \pbox{1.5 in}{\InPlace} & \pbox{1.5 in}{\GroupCommit} \\
    \emph{Software buffer} & \cellcolor[gray]{.8}\pbox{1.1 in}{Traditional WAL/ARIES} & \cellcolor[gray]{.8}\pbox{1.5 in}{Updates both buffer and NVRAM} & \cellcolor[gray]{.8}\pbox{1.5 in}{Buffer limits batch size} \\
    \emph{Hardware buffer} & \cellcolor[gray]{.95}\pbox{1.1 in}{Impractical} & \cellcolor[gray]{.95}\pbox{1.5 in}{Slow uncached NV\-RAM reads} & \cellcolor[gray]{.95}\pbox{1.5 in}{Requires hardware support} \\
    \emph{Replicate to DRAM} & \multicolumn{3}{l}{\pbox{4.1 in}{\cellcolor[gray]{.8}Provides fast reads and removes buffer management, but requires large DRAM capacity}} \\
  \end{tabular*}
  \caption{\textbf{NVRAM design space.} Database designs include recovery mechanisms (top) and cache configurations (left).}
  \label{table::DesignSpace}
\end{table*}
