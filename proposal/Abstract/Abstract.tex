Emerging storage technologies offer an alternative to disk that is durable and allows faster data access.
Flash memory, made popular by mobile devices, provides block access with low latency random reads.
New nonvolatile memories (NVRAM) are expected in upcoming years, presenting DRAM-like performance alongside persistent storage.
Wheres both technologies accelerate data accesses due to increased raw speed, used merely as disk replacements they may fail to achieve their full potentials.
Flash's asymmetric read/write access (i.e., reads execute faster than writes) opens new opportunities to optimize Flash-specific access.
Similarly, NVRAM's low latency persistent accesses allow new designs for high performance failure-resistant applications.

This thesis addresses software and hardware system design for such storage technologies.
First, I investigate analytics query optimization for Flash, expecting Flash's fast random access to require new query planning.
While intuition suggests scan and join selection should shift between disk and Flash, I find that query plans chosen assuming disk are already near-optimal for Flash.
Second, I examine new opportunities for durable, recoverable transaction processing with NVRAM.
Existing disk-based recovery mechanisms impose large software overheads, yet updating data in-place requires frequent device synchronization that limits throughput.
I introduce a new design, \GroupCommit, to amortize synchronization delays over many transactions, increasing throughput at some cost to transaction latency.
Finally, I propose to research programming interfaces and hardware designs to enable intuitive, high performance recoverable data structures with NVRAM, extending memory consistency with persistent semantics to introduce \emph{Persistent Memory Consistency}.
