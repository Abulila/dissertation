Emerging storage technologies offer an alternative to disk that is durable and allows faster data access.
Flash memory, made popular by mobile devices, provides block access with low latency random reads.
Additionally, new nonvolatile memories (NVRAM) are expected in upcoming years, presenting DRAM-like performance alongside persistent storage.
Wheres both technologies accelerate data access due to their raw speed, used merely as a disk replacement they may fail to achieve their full potentials.
Flash's asymmetric read/write access (i.e., reads execute faster than writes) opens new opportunities to optimize flash-specific access.
Similarly, NVRAM's new persistent interface allows new desings for high performance failure-resistant transaction processing.

This thesis addresses software and hardware system design for new and upcoming storage technologies.
First, I investigate analytics query optimization for flash, expecting flash's fast random access to require new query planning.
While flash allows some queries to benefit from alternative query optimization, the range of queries affected and relative performance increase is negligible.
Second, I examine new opportunities for durable, recoverable transaction processing with new NVRAM storage.
Existing disk-based recovery mechanisms impose large software overheads, yet updating data in-place requires frequent device synchronization that limits throughput.
Subsequently, I introduce a new design, \GroupCommit, to amortize synchronization delays over many transactions, increasing throughput at some cost to transaction latency.
Finally, I propose to research hardware designs and programmer interfaces to provide high performance recoverable data structures with NVRAM, extending memory consistency with persistent semantics to introduce \emph{Persistent Memory Consistency}.
