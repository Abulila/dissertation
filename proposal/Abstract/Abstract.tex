This template conforms to University of Michigan abstract and dissertation format guidelines as of September 2008. It is an update to a template that has been floating around among grad students here for about 20 years. The main components are the thesis.tex file and the rac.sty file, the latter of which should not need any modification. If BibTeX is used (and for a dissertation, it should be), then References.bib is also needed. If a list of acronyms is desired, make all additions in abbr.tex and read acronym.pdf on ctan.org for details on how to call them in the text. Other files in this template that may be helpful, but don't necessarily need to be used include a style file that formats your bibliography in AGU format (agu04.bst) and a style file that allows you to use abbreviations for journal names (aas\_macros.sty) when typing out the bibliography. This will be necessary if you grab BibTeX information from places like the NASA ADS, which sometimes uses journal name abbreviations. It is useful to separate chapters into their own subfolders, with each folder containing the chapter's .tex file as well as all associated figures. For the figures, just call the name of the file, without the suffix (i.e., includegraphics\{Chap5/LabSetup\}) and the graphicx package will figure out what type of file it is. To compile to pdf, some format other than .eps must be used with the figures. To compile to ps, the figures need to be in ps or eps. If using \LaTeX\, in a Windows environment, there are several different editors and programs that can be used. One set that is known to work well is the following, which can each be found with a simple web search and which should be installed in this order: a Perl distribution such as ActivePerl, a \TeX\, distribution such as MikTex, a \LaTeX\, editor such as TeXnicCenter, a postscript interpreter like Ghostscript, and a postscript viewer like GSView. I included a few pages of sample code in chapter 2 to help you get started, including code for writing equations, citations, abbreviations, tables, and calling graphics. Always be sure to compile your thesis.tex file a couple of times to get the references and page numbering updated. Good luck. -jg