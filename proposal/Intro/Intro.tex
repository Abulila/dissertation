For decades disk has been the primary technology for durable and large-capacity storage.
Although inexpensive and dense, disk provides high performance only for coarse-grained sequential access and suffers enormous slowdowns for random reads and writes.
Recently, several new technologies have emerged as popular or viable storage alternatives.
Flash memory, primarily used for mobile storage, has gained traction as a high-performance enterprise storage solution.
Nonvolatile Random Access Memories (NVRAM), such as phase change memory and spin-transfer torque RAM, have emerged as high performance storage alternatives \cite{BurrKurdi08}.

These technologies offer significant performance improvements over disk, while still providing durability with great storage capacity.
As drop-in replacements for disk, Flash and NVRAM greatly accelerate storage access.
However, the disk interface fails to leverage specific device characteristics.
Section~\ref{sec:Background:Storage} provides a background on these storage technologies and specifically how their performance differs from disk.

This dissertation investigates how several data-centric workloads interact with future storage technologies, the relevant software and algorithms, and in some instances computer hardware.
Specifically, I consider analytics (commonly Decision Support Systems -- DSS -- popular in ``Big Data") and On-Line Transaction Processing (OLTP).
Both workload classes have been optimized to surmount disk's constraints, yet storage devices often remain the performance bottleneck and dominant cost.
I match each workload to the emerging storage technology that suits it best and address specific opportunities or deficiencies in the software and hardware systems.

\section{Analytics}
\label{sec:Intro:Analytics}

Analytics relies on disk to provide enormous data capacity.
Typical analytics work-flow involves taking a snapshot of data from an online database and mining this data for complex, yet useful, patterns.
While applications do not rely on disk's durability for recovery (in fact, instances that fit in main memory have no need for disk), modern analytics data sets reach peta-byte scale \cite{Economist10}, and accessing such large data imposes the dominant bottleneck.
Such capacity is currently achieved only by dense disk and Flash memory.

Decades of research have provided modern analytics databases with tools to minimize storage accesses, particularly slow random accesses (e.g., disk-specific indexes, join algorithms to minimize page access and produce large sequential runs).
Whereas these optimizations are still effective for Flash, they fail to leverage Flash's ability to quickly read non-sequential data (many optimizations purposefully avoid random access patterns on disk).
As examples, I consider access paths (various scan types) and join algorithms.
An historic rule of thumb for scans is that an index should be used when less than 10\% of rows in a table are returned, otherwise the entire table should be scanned \cite{RamakrishnanAndGehrke}.
The intuition is that locating rows from an index requires random reads as well as reading additional pages from the index itself.
At sufficiently high selectivities, accessing the entire table, scanning all rows and returning those that satisfy the query, provides a faster access path.
One would expect this selectivity (10\%) to increase when replacing disk with Flash -- Flash is no longer penalized by random reads, preferring any scan that minimizes total page accesses.
Similarly, different ad hoc join algorithms (those that do not use indexes: block-nested loops, sort-merge join, and hybrid-hash join) present different storage access patterns and may be variably suited to disk and Flash.
These algorithms and query optimization are further discussed in Section~\ref{sec:Background:Scans}.

My results, originally presented in ADMS 2010 \cite{PelleyWenisch11} and discussed in Chapter~\ref{chap:FlashOpti}, show that while both previous hypotheses are correct, their significance is negligible.
Optimal access path (index vs. table scan) only changes between disk and Flash for a small range of query selectivities, and queries within that range see only a small performance improvement.
Additionally, join algorithm choice makes little difference, as optimized join algorithms already minimize storage accesses, regardless of access pattern---join algorithms optimized for disk are already optimized for Flash.
I conclude that the page-oriented nature of Flash limits further analytics-Flash optimization.
On the other hand, emerging byte-addressable NVRAMs offer finer-grained access.
However, analytics does not require persistent storage, instead using NVRAM as a replacement for DRAM.
As DRAM-resident analytics techniques are already well established, I instead investigate using NVRAM persistence specifically to provide failure recovery, supporting durable transactions.

\section{Transaction Processing}
\label{sec:Intro:OLTP}

Databases have been designed for decades to provide high-throughput transaction processing with disk.
Write Ahead Logging (WAL) techniques, such as ARIES \cite{MohanHaderle92}, transform random writes into sequential writes and minimize transactions' dependences on disk accesses.
Section~\ref{sec:Background:Recovery} outlines modern recovery management, focusing on ARIES.
With sufficient device throughput (IOPS) and read-buffering, databases can be made compute-bound and recover near-instantly.
NVRAMs provide this massive storage throughput to the masses.

Whereas ARIES is necessary for disk, it presents only unnecessary software overheads to NVRAM.
I show that removing ARIES improves transaction throughput by alleviating software bottlenecks inherent in centralized logging.
Instead, NVRAM allows data to be updated in-place, enforcing data persistence immediately and providing correct recovery via transaction-local undo logs.

NVRAMs, however, are not without their limitations.
Se\-veral candidate NVRAM technologies exhibit larger read latency and significantly larger write latency compared to DRAM \cite{BurrKurdi08}.
Additionally, whereas DRAM writes benefit from caching and typically are not on applications' critical paths, NVRAM writes must become persistent in a constrained order to ensure correct recovery.
I consider an NVRAM access model where correct ordering of persistent writes is enforced via \emph{persist barriers}, which stall until preceding NVRAM writes complete; such persist barriers introduce substantial delays when NVRAM writes are slow.

To address these challenges I investigate accelerating NVRAM reads with various cache architectures and capacities, and avoid persist barrier delays by introducing a new recovery mechanism, \GroupCommit.
Database designs are discussed in Chapter~\ref{chap:OLTP_design}.
As expected, low latency memory-bus-connected NVRAM needs little additional caching (on-chip caches suffice) and updating data in-place is a simple and viable recovery strategy.
However, long latency NVRAM and complex interconnects (e.g., Non-Uniform Memory Architectures -- NUMA, PCIe-attached NVRAM, or distributed storage) benefit from DRAM caching and \GroupCommit, improving throughput.
I investigate specifically how NVRAM read and persist barrier latencies drive OLTP system design.
These results and additional evaluations are presented in Chapter~\ref{chap:OLTP_eval}.
This work was originally published at VLDB \cite{Pelley13}.

\section{Memory Persistency}
\label{sec:Intro:PMC}

The previous work looks at how OLTP recovery mechanisms should be designed, considering only the average delay incurred by persist barriers.
The final portion of my dissertation investigates specific programming interfaces to order NVRAM writes.
Whereas existing memory consistency models provide control over the order and visibility of volatile memory reads and writes across threads, there are no equivalent models to reason about data persistence.
Memory consistency may be relaxed, allowing communicating threads to each observe different memory read and write orders.
Such memory consistency models improve performance, but require complex reasoning and additional programming mechanisms (memory barriers) to ensure expected behavior.
Memory models are described in Section~\ref{sec:Background:MemoryConsistency}.

Similarly, NVRAM write order may be relaxed, improving performance by allowing writes to occur in parallel or out of order while still ensuring correct recovery.
I introduce \emph{memory persistency}, a framework that extends memory consistency, to reason about NVRAM write order.
Relaxed memory persistency models use persist barriers to enforce specific write orders, guaranteeing that data is correctly recovered after failure.
I define memory persistency and enumerate the possible design space in Chapter~\ref{chap:Persistency}.
Interestingly, memory persistency models may be de-coupled from the underlying memory consistency models, separately enforcing the order in which writes become durable and the order in which writes become visible to other threads.
I introduce a persistent queue benchmark and several memory persistency models in Chapter~\ref{chap:PersistencyModels}.
Finally, I evaluate these models in Chapter~\ref{chap:PersistencyEval}.
Strict persistency models slow execution nearly $30\times$ relative to existing throughput on volatile, nonrecoverable systems.
Relaxed persistency models regain lost throughput by improving NVRAM write concurrency.

\section{Data Center Infrastructure}
\label{sec:Intro:Additional}
The themes of this dissertation include performance, cost efficiency, and reliability.
While I focus on storage architectures, I additionally published work regarding the cost and reliability of data center infrastructure.
Appendix~\ref{app:WEED} contains ``Understanding and Abstracting Total Data Center Power," published at WEED 2009 \cite{PelleyMeisner09}.
This work presents power/energy models for all aspects of the data center, including power distribution, battery backups, cooling infrastructure, and IT equipment.
Appendix~\ref{app:PowerRouting} contains ``PowerRouting: Dynamic Power Provisioning for the Data Center," published at ASPLOS 2010 \cite{PelleyMeisner10}.
PowerRouting spreads power distribution responsibility throughout the data center to minimize installed power infrastructure capacity while maintaining reliability, minimizing data center cost.
The key insight is that data centers typically over-provision infrastructure, resulting in under-utilized (and often unnecessary) equipment.
PowerRouting leverages compute-specific knowledge of the IT workload to more effectively utilize power infrastructure.
Both of these works are included without modification.

During my investigations I discovered that, in many regards, industry is ahead of academia at decreasing operating costs and improving infrastructure efficiency.
As such, the opportunity to contribute meaningful new techniques to improve infrastructure is rapidly diminishing.
Recognizing storage and memory as primary concerns for energy efficiency, reliability, and cost, I focus on new and emerging storage technologies in this dissertation.

\section{Summary}
\label{sec:Intro:Summary}
This dissertation investigates new techniques for accelerating data access for new NVRAM storage technologies, and is organized as follows:
Chapter~\ref{chap:Background} contains background information on storage technologies, database optimizations, and memory consistency.
This background forms the foundation of the work that follows.
Chapter~\ref{chap:FlashOpti} considers taking advantage of Flash's fast random reads to accelerate database analytics.
In Chapter~\ref{chap:OLTP_design}, I describe potential software designs for OLTP using NVRAM.
Chapter~\ref{chap:OLTP_eval} details a methodology to evaluate NVRAM (devices not readily available) on modern hardware and evaluates several aspects of OLTP running on NVRAM.
Chapter~\ref{chap:Persistency} introduces and defines memory persistency.
Specific memory persistency models and a persistent queue data structure are proposed in Chapter~\ref{chap:PersistencyModels}.
Finally, I evaluate these memory persistency models in Chapter~\ref{chap:PersistencyEval}.
