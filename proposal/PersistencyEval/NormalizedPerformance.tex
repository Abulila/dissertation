\begin{table*}
  \centering
  \begin{tabular}{ l l l l l l l l l l }
    \hline
    Threads & \multicolumn{3}{c}{Copy While Locked} & \multicolumn{3}{c}{Two-Lock Concurrent} & \multicolumn{3}{c}{Queue Holes} \\
    & Strict & Epoch & Strand & Strict & Epoch & Strand & Strict & Epoch & Strand \\
    \hline \hline
    %1 & 0.03402668948225257 & 0.17329066524439704 & 12.395878787878786 & 0.07956161591630416 & 0.5569551810923878 & 28.931290043290044 & 0.03243921666510665 & 0.17329066524439704 & 13.36387317909169 \\
    %8 & 0.05787677206992028 & 3.167189491481337 & 21.084432900432898 & 0.4319337035691424 & 3.3529680365296803 & 21.615757575757577 & 0.2539605628870943 & 1.9442439473904665 & 19.535723905723902 \\
    1 & 0.034 & 0.17 & \textbf{12} & 0.080 & 0.56 & \textbf{29} & 0.032 & 0.17 & \textbf{13} \\
    8 & 0.058 & \textbf{3.2} & \textbf{21} & 0.43 & \textbf{3.4} & \textbf{22} & 0.25 & \textbf{1.9} & \textbf{20} \\
    \hline
  \end{tabular}
  \caption{
    \textbf{Relaxed Persistency Performance.}
    Persist-bound insert rate normalized to instruction execution rate assuming 500ns persist latency.
    System throughput is limited by the lower of persist and instruction rates---at greater than 1 (bold) instruction rate limits throughput; at lower than 1 execution is limited by the rate of persists.
    While strict persistency limits throughput, epoch persistency maximizes performance for many threads and strand persistency is necessary to maximize performance with one thread.
  }
  \label{table::RelaxedPerformance}
\end{table*}
