\section{Conclusion}
\label{section::conclusion}

The capital cost of power delivery infrastructure is one of the largest components of data center cost, rivaling energy costs over the life of the facility.  In many data centers, expansion is limited because available power capacity is exhausted.  To extract the most value out of their infrastructure, data center operators over-subscribe the power delivery system.  As long as individual servers connected to the same PDU do not reach peak utilization simultaneously, over-subscribing is effective in improving power infrastructure utilization.  However, coordinated utilization spikes do occur, particularly among collocated machines, which can lead to substantial throttling even when the data center as a whole has spare capacity.

In this paper, we introduced a pair of complementary mechanisms, shuffled power distribution topologies and \PowerRouting, that reduce performance throttling and allow cheaper capital infrastructure to achieve the same performance levels as current data center designs.  Shuffled topologies permute power feeds to create strongly-connected topologies that reduce reserve capacity requirements by spreading responsibility for fault tolerance.  \PowerRouting schedules loads across redundant power delivery paths to shift power delivery slack to satisfy localized utilization spikes.  Together, these mechanisms reduce capital costs by 32\% relative to a baseline high-availability design when provisioning for zero performance throttling. Furthermore, with energy-proportional servers, the power capacity reduction increases to 47\%.
